% ----------------------------------------------------------------------
%  Základní nastavení dokumentu - musí být na začátku každého tex
%  (pořadí příkazů v této části je důležité!)
% ----------------------------------------------------------------------

%  Typ dokumentu - článek, prezentace aj.
% Standardní základ
%\documentclass[a4paper,10pt]{article} 
%\usepackage[letterpaper]{geometry}
%\geometry{verbose,tmargin=1.5cm,bmargin=2cm,lmargin=2cm,rmargin=2cm}

% Alternativní verze - vhodnější formát papíru (větší hustota textu)
\documentclass[10pt]{scrartcl}
\KOMAoptions{DIV=20} % formát papíru a odsazení od jeho okrajů

%  Kódování výstupu - aby šlo z pdf kopírovat včetně háčků a čárek
\usepackage[T1]{fontenc} 

%  Kódování vstupu - v kódu lze použít háčky a čárky
\usepackage[utf8]{inputenc} 
\usepackage[version=3]{mhchem}
%  Základní typografická pravidla češtiny/slovenštiny
\usepackage[czech]{babel} 
%\usepackage[slovak]{babel}
% použijte jen jeden z příkazů
% \usepackage{mathrsfs}

%  Lépe vypadající písmo pro T1 kódování
\usepackage{lmodern} 
% je možno zakomentovat, nepoužívá-li se T1 kódování

\usepackage{adjustbox}
\usepackage{float} % rozšířené možnosti umístění obrázků
\usepackage{soul}
\usepackage{xargs}
\usepackage{makecell}
% \usepackage{fkssugar}
%  Formátování stránek, empty = odstraní číslování
% \pagestyle{empty}

%  Řádkování
\linespread{1.1}


% ----------------------------------------------------------------------
%  Doplňující balíčky
% ----------------------------------------------------------------------

%  Po desetinné čárce v matematickém módu se nevytvoří mezera
\usepackage{icomma} 

%  Umožňuje pracovat s grafikou
\usepackage{graphicx}

%  Umožňuje použít dva obrázky vedle sebe
\usepackage{subcaption}

%  Pro vkládání obrázků ve formátu eps (např. z gnuplot)
\usepackage{epstopdf} 

%  Automaticky odsadí i první paragraf v každé sekci
\usepackage{indentfirst}

%  Umožňuje rozdělovat obsah na více sloupců
\usepackage{multicol}
\usepackage{booktabs}
\usepackage{pgffor}

%  Umožňuje používat hypertextové odkazy, nastavuje jejich vlastnosti
\usepackage[unicode]{hyperref}
\usepackage{paralist}
\usepackage{wasysym}
% ----------------------------------------------------------------------
%  Matematika
% ----------------------------------------------------------------------

%  Lepší zobrazování matematiky (rozšíření sum o \limits atd.)
\everymath{\displaystyle}

%  Široké spektrum příkazů pro matematiku
% (Umožní např. psát přes \mathbb{N/R/Q/..} množiny čísel)
\usepackage{amsmath,amssymb}

%  Velikost fontu matematických výrazů v dokumentu lze pro danou
% základního fontu dokumentu upravit pomocí:
% \DeclareMathSizes{X}{Y}{Z}{U} kde:
% X je velikost fontu v dokumentu, pro kterou se matematika upraví
% Y je standartní velikost fontu matematiky
% Z je velikost fontu zmenšených (vnořených výrazů)
% U je velikost fontu ještě více zmenšených (vnořených výrazů).
\DeclareMathSizes{10}{10}{8}{7}

%  Široké spektrum příkazů pro fyziku
\usepackage{physics}

%  Psaní SI jednotek
\usepackage[separate-uncertainty=true]{siunitx}
\sisetup{output-decimal-marker = {,}}


\usepackage{comment}


%  Nám bližší zápis písmene epsilon
\AtBeginDocument{%
%\let\phi\varphi
\let\epsilon\varepsilon
}


% ----------------------------------------------------------------------
%  Pro češtinu/slovenštinu
% ----------------------------------------------------------------------

%  Lokalizace některých názvů do češtiny/slovenštiny
\addto\captionsczech{\renewcommand{\figurename}{Obr.}}
\addto\captionsczech{\renewcommand{\tablename}{Tab.}}
%\addto\captionsczech{\renewcommand{\refname}{Reference}}

\addto\captionsslovak{\renewcommand{\figurename}{Obr.}}
\addto\captionsslovak{\renewcommand{\tablename}{Tab.}}
%\addto\captionsslovak{\renewcommand{\refname}{Reference}

%===========================Fykosi===========

% \newcommand\tg{\mathop\mathrm{tg}\nolimits}	% cesky tangens
%   \renewcommand\tan{\mathop\mathrm{tg}\nolimits}
%   
 \renewcommand\d{\mathrm{d}}			% differential sign 
  
%%%%%%%%
%%%%%%%%%%%FYKOSi %%%%%%%%%%%%%
% \newcommand\my@jd[1]{\begingroup\begingroup\!\jed@B#1@@}
% \newcommand\jd[1]{\ensuremath{\my@jd{#1}}} % pro psani samostatnych jednotek (napr. $\jd{m.s^{-1}}$)
% \newcommand\popi[2]{{$\dfrac{#1}{\jd{#2}}$}} % labels for graphs/tables
\newcommand{\ti}{\textit} % zkrácený příkaz pro kurzívu
\newcommand{\tb}{\textbf} % zkrácený příkaz pro tučné písmo
\newcommand{\bv}{\mathbf} % zkrácený příkaz pro tučné písmo

\let\underscore\_                                           % underscore sign
\let\xor\^                                                  % xor sign
\renewcommand\_[2][1]{\ifmmode _{\textnormal{\scalebox{#1}{#2}}}\else\underscore#2\fi} %roman subscript
\renewcommand\^[2][1]{\ifmmode ^{\textnormal{\scalebox{#1}{#2}}}\else\xor#2\fi} %roman superscript
\def\smallind{0.8}

\def\({\left(} \def\){\right)}                              % parentheses

  %================================
\newcommand{\pic}[3]{%
\begin{figure}[h]
\centering
\includegraphics[width=0.8\textwidth]{./img/#1}
\caption{#3}
\label{#2}
\end{figure}
}
  
\newcommand{\footnotei}[2]{%
\mbox{%
\setbox0\hbox{#1}%
\copy0%
\hspace{-\wd0}}%
\footnote{#2}%
}
% Odkomentujte následující příkaz, máte-li stažený balíček encxvlna 
% (nutno stáhnout manuálně)
%\usepackage{encxvlna} %vloží nezlomitelné mezery k jednopísmenným

% \begingroup\uccode`~=`e\uppercase{\endgroup\let~}\E   % aktivni e se chova jako \E
% \mathchardef\mathquote=\mathcode`\"
% \def\jed@A #1 {\begingroup\def~{\,}\mathcode`\.="013B\mathcode`\,="013B \begingroup \mathcode`\e="8000 #1 \jed@B} % numbers written with dot or comma
% \def\jed@B #1@@{\endgroup\if @#1@\else\,\mathcode`\.="0201 \mathup{#1}\fi\endgroup}
% \begingroup\uccode`~=`"\uppercase{\endgroup\def~}#1"{\jed@A #1 @@}
% \mathcode`\"="8000
% 
% \@ifpackageloaded{babel}{%
%   \PackageWarning{fkssugar}{Package babel detected, macros with active double quote ", will be disabled.}
% % -- Uncomment if you need use standard babel double quote macros --
% %  \begingroup\uccode`~=`*\uppercase{\endgroup\def~}#1*{\jed@A #1 @@}
% %  \mathcode`\*="8000
%   \catcode`\"=12
% }{%
% }}



\newcommandx{\eq}[1]{
        \begin{equation}
                #1
        \end{equation}
}

\newcommandx{\cit}[1]{
        \begin{compactitem}
                #1
        \end{compactitem}
}

\newcommandx{\cen}[1]{
        \begin{compactenum}
                #1
        \end{compactenum}
}

\renewcommand\C{^\circ\mskip-2mu\mathrm{C}}                  % Celsiuv stupen
% ----------------------------------------------------------------------
%  Soubor s makry
% ----------------------------------------------------------------------
% ----------------------------------------------------------------------
%  Identifikace protokolu (příkazy lze použít v celém dokumentu)
% ----------------------------------------------------------------------

%  Nastaví autora, název, datum, skupinu měření apod. 
\newcommand{\Institute}{FJFI~ČVUT~v~Praze}

% Examples -- subject
%\newcommand{\Subject}{Základy fyzikálních měření}
%\newcommand{\Subject}{Experimentální reaktorová fyzika}  %odkomentujte dle potřeby

%%%%%%%%%%%%%%%%%%%%%%%%%%%%%


%  Máte-li více spoluměřících než jednoho, vložte jen jejich příjmení
\newcommand{\Author}{}
\newcommand{\Coauthor}{} 
\newcommand{\Group}{} %číslo skupiny v rámci praktika, nikoli kruh
\newcommand{\Labdate}{} %den, kdy chodíte na praktika, nikoli obor

%  Tato část bude v každém protokolu jiná, nezapomeňte upravit!
\newcommand{\Title}{}
% \newcommand{\Labdate}{ -- } %datum měření, nikoli datum odevzdání

% I guess that is not the value one would like to share
% \newcommand{\Worktime}{5 h} %jak dlouho vám trvalo vypracování protokolu




% ----------------------------------------------------------------------
%  Vlastní příkazy
% ----------------------------------------------------------------------


%  Matematika
\newcommand{\ee}{\mathrm{e}} %eulerovo číslo
\newcommand{\ii}{\mathrm{i}} %imaginární jednotka

% Jednotky
\newcommand{\unit}[1]{\,\mathrm{#1}} %jednotky zadávejte pomocí tohoto příkazu
\renewcommand{\deg}{\ensuremath{\mathring{\;}}} %symbol stupně
\newcommand{\celsius}{\ensuremath{\deg\mathrm{C}}} %stupně celsia

%(hodnota plus mínus chyba) jednotka
\newcommand{\hodn}[3]{(#1 \pm #2)\unit{#3}} 

%veličina [jednotka] do hlavičky tabulky
\newcommand{\tabh}[2]{\ensuremath{#1\,[\mathrm{#2}]}} 




%  nachází se ve složce /tex/



% ----------------------------------------------------------------------
%  Nastavení odkazů a výsledného pdf
% ----------------------------------------------------------------------
\hypersetup{
colorlinks=true, 
citecolor=blue, 
filecolor=blue, 
linkcolor=blue,
urlcolor=blue, 
pdftitle={\Title},    % title
pdfauthor={\Author},     % author
pdfsubject={Protokol},   % subject of the document
pdfcreator={\Author},   % creator of the document
%     pdfproducer={Producer}, % producer of the document
%     pdfkeywords={keywords}, % list of keywords
pdfnewwindow=true,      % links in new window
}

% ----------------------------------------------------------------------
%  Začátek dokumentu - formátování na výstup
% ----------------------------------------------------------------------
\begin{document}

% Interní proměnné, možno zobrazovat u prezentací, používají se při
% generování pomocí \titlepage apod.
\author{\Author}
\title{\Title}
\date{\Labdate}


% ----------------------------------------------------------------------
%  Hlavička dokumentu
% ----------------------------------------------------------------------

\setlength{\parindent}{0cm}
\begin{multicols}{2}
\textsf{\textbf{\Subject \hspace{9.5cm} \Institute}\\
% \textsf{\textbf{\Subject \hfill \Institute}\\
%\large  \Title \\[0.5cm]
\textbf{\large{\Title}}}

\begin{tabular}{rlrl}
	 \textsf{Jméno:} & \textbf{\textsf\Author}    &      \textsf{Zpracováno:} &
   \textsf{\Labdate} \\[1.5pt]
	  \textsf{Měřeno:} & \textbf{\textsf{-}}     & \textsf{Klasifikace.:} &    \\[1.5pt]
	%\textsf{Měřeno:} & \textbf{\textsf{\Labdate}} 
\end{tabular}

\begin{flushright}

\includegraphics[scale=0.15]{img/fjfi.pdf}
\hspace{0.4cm}
\includegraphics[scale=0.15]{img/cvut.pdf}


 

\end{flushright}
\end{multicols}

\hrule

% ----------------------------------------------------------------------
%  Tělo dokumentu
% ----------------------------------------------------------------------

\setlength{\parindent}{0.5cm}

% ----------------------------------------------------------------------
%  Protokol

\pagenumbering{arabic}  % číslování stránek čísly
% ----------------------------------------------------------------------
%  Pracovní úkoly
% ----------------------------------------------------------------------
\section{Pracovní úkoly}

\begin{compactenum}
  % Examples
  % \item Measure sth
  % \item etc
\end{compactenum}
%
% ----------------------------------------------------------------------
%  Použité přístroje a pomůcky
% ----------------------------------------------------------------------
\section{Použité přístroje a pomůcky}

%
% ----------------------------------------------------------------------
%  Teoretický úvod
% ----------------------------------------------------------------------
\section{Teoretický úvod}
% \subsection{}
% 
% Příklad chemické či jaderné reakce
% \eq{
%   \ce{^{2}H + ^{2}H -> ^{3}He + ^{1}n}\, Q = \SI{3,27}{MeV}\,.
% }
% Příklad odkazu na obrázek~\ref{fig:ddSpectrum}. 
% Příklad použití balíčku \texttt{siunitx} je maximální výtěžek \SI{7e6}{s^{-1}},
% za kterým použijeme \texttt{footnote} nad interpunkcí\footnotei{.}{Hodnota
% používaná na KJR.} 
% Tu vložíme obrázek pomocí \texttt{zjednodušené syntaxe}. 
% \pic{imageNameInIMGdir.png}{fig:myLabel}{
%   Awesome and looong caption.
% }
% Isotope \ce{^{2}H} 
% 
% Align environment to center set of equations
% \begin{align}
% 	\frac{\mathrm{d} n(t)}{\mathrm{d}t} &= \frac{k_{ef}(1-\beta\_{ef})-1}{\ell} n(t) + \overline{\lambda}c(t) + S \,,\label{n_dest}\\ 
% 	\frac{\mathrm{d} c(t)}{\mathrm{d} t} &= \frac{\beta\_{ef}k_{ef}}{\ell} n(t) - \overline{\lambda} c(t)\,,\label{c_dest}
% \end{align}
% a \uv{něco v českých uvozovkách} 
% 
% Siplified equations' syntax
% \eq{
% 	0 = \frac{k_{ef}(1-\beta\_{ef})-1}{\ell} n(t) + \,,
% }
% 
% 
%
%-------------------------------------
%Postup měření
%--------------------------------------
\section{Postup měření}



%
%---------------------------------------------
%Zpracování
%------------------------------------------------
\section{Zpracování}

% Example table
% \begin{table}[h] % consider using H if you wanted fixed position
% \centering
%   \caption{}
% \begin{tabular}{Scccc} % S from siunitx to center decimal point
% 	\toprule
%   {$U [\si{kV}]$} &{$N(t) [-]$} & {$t [\si{s}]$} & {$CR [\si{s^{-1}}]$} &
%   {$\sigma_{CR} [\si{s^{-1}}]$}\\
%   \midrule
% 90.985332&26842&60&450&20\\
% 95.985332&31166&60&520&20\\
% 10.9853320&36375&60&610&20\\
% 10.9853325&41887&60&700&30\\
% 11.9853320&34028&43&790&30\\
% 11.9853325&52933&60&880&30\\
% 12.9853320&58753&60&980&30\\
%   \bottomrule
% \end{tabular}
% \label{tab:mylabel}
% \end{table}
% 
% Classical figure instead of \texttt{pic} env.
% \begin{figure}[h]
%   \centering
%   \caption{}
%   \resizebox{!}{0.5\textwidth}{\input{name_of_gnuplot out file}} % for those
% %   using gnuplot eps terminal output
% \end{figure}
% 
% \texttt{eqref} for nice eq. referencing  \eqref{sJInt}
% An example of \texttt{siunitx} package 
% for adding tolerances $A = \SI{12.8(4)}{s}$


\section{Diskuse}

%
% ----------------------------------------------------------------------
%  Závěr
% ----------------------------------------------------------------------
			
\section{Závěr}

% ----------------------------------------------------------------------

% ----------------------------------------------------------------------
%  Literatura

\section{Použitá literatura}		
\begingroup
\renewcommand{\section}[2]{}
% ----------------------------------------------------------------------
%  Reference
% ----------------------------------------------------------------------

% sem doplňujte použité zdroje
% \begin{thebibliography}{9}
%     \bibitem{berkeley} LOU, Tak Pui. Compact D-D/D-T Neutron Generators and
%       Their Applications. Berkeley, CA, USA, 2003. Dostupné také z:
%       \url{https://cds.cern.ch/record/744955/files/34065517.pdf}. Disertace. University of California, Berkeley.
%     \bibitem{thermoScientific} P 385 Neutron Generator. ThermoFisher Scientific
%       [online]. 2021 [cit. 2021-02-07]. Dostupné z:
%       \url{https://www.thermofisher.com/order/catalog/product/10135952?fbclid=IwAR2PispEfS7iXxkdSCHvye2Y9ctbk-ALrTerMCvve3HtNeVstf6YCuM6JUg#/10135952}

\end{thebibliography}

\endgroup
% ----------------------------------------------------------------------


% ----------------------------------------------------------------------
%  Příloha

% \clearpage 
%%  \pagenumbering{roman}  % číslování stránek písmeny
% 
% \setcounter{equation}{0}
% \setcounter{section}{0}
% \numberwithin{equation}{section} % případné rovnice budou číslované pod číslem kapitoly
% 
% \part*{\LARGE{Příloha}}
% 
\input{tex/apendix}
% ----------------------------------------------------------------------



\end{document}

% ----------------------------------------------------------------------
%  Konec dokumentu
% ----------------------------------------------------------------------
