% ----------------------------------------------------------------------
%  Pracovní úkoly
% ----------------------------------------------------------------------
\section{Pracovní úkoly}

\begin{compactenum}
  % Examples
  % \item Measure sth
  % \item etc
\end{compactenum}
%
% ----------------------------------------------------------------------
%  Použité přístroje a pomůcky
% ----------------------------------------------------------------------
\section{Použité přístroje a pomůcky}

%
% ----------------------------------------------------------------------
%  Teoretický úvod
% ----------------------------------------------------------------------
\section{Teoretický úvod}
% \subsection{}
% 
% Příklad chemické či jaderné reakce
% \eq{
%   \ce{^{2}H + ^{2}H -> ^{3}He + ^{1}n}\, Q = \SI{3,27}{MeV}\,.
% }
% Příklad odkazu na obrázek~\ref{fig:ddSpectrum}. 
% Příklad použití balíčku \texttt{siunitx} je maximální výtěžek \SI{7e6}{s^{-1}},
% za kterým použijeme \texttt{footnote} nad interpunkcí\footnotei{.}{Hodnota
% používaná na KJR.} 
% Tu vložíme obrázek pomocí \texttt{zjednodušené syntaxe}. 
% \pic{imageNameInIMGdir.png}{fig:myLabel}{
%   Awesome and looong caption.
% }
% Isotope \ce{^{2}H} 
% 
% Align environment to center set of equations
% \begin{align}
% 	\frac{\mathrm{d} n(t)}{\mathrm{d}t} &= \frac{k_{ef}(1-\beta\_{ef})-1}{\ell} n(t) + \overline{\lambda}c(t) + S \,,\label{n_dest}\\ 
% 	\frac{\mathrm{d} c(t)}{\mathrm{d} t} &= \frac{\beta\_{ef}k_{ef}}{\ell} n(t) - \overline{\lambda} c(t)\,,\label{c_dest}
% \end{align}
% a \uv{něco v českých uvozovkách} 
% 
% Siplified equations' syntax
% \eq{
% 	0 = \frac{k_{ef}(1-\beta\_{ef})-1}{\ell} n(t) + \,,
% }
% 
% 
%
%-------------------------------------
%Postup měření
%--------------------------------------
\section{Postup měření}



%
%---------------------------------------------
%Zpracování
%------------------------------------------------
\section{Zpracování}

% Example table
% \begin{table}[h] % consider using H if you wanted fixed position
% \centering
%   \caption{}
% \begin{tabular}{Scccc} % S from siunitx to center decimal point
% 	\toprule
%   {$U [\si{kV}]$} &{$N(t) [-]$} & {$t [\si{s}]$} & {$CR [\si{s^{-1}}]$} &
%   {$\sigma_{CR} [\si{s^{-1}}]$}\\
%   \midrule
% 90.985332&26842&60&450&20\\
% 95.985332&31166&60&520&20\\
% 10.9853320&36375&60&610&20\\
% 10.9853325&41887&60&700&30\\
% 11.9853320&34028&43&790&30\\
% 11.9853325&52933&60&880&30\\
% 12.9853320&58753&60&980&30\\
%   \bottomrule
% \end{tabular}
% \label{tab:mylabel}
% \end{table}
% 
% Classical figure instead of \texttt{pic} env.
% \begin{figure}[h]
%   \centering
%   \caption{}
%   \resizebox{!}{0.5\textwidth}{\input{name_of_gnuplot out file}} % for those
% %   using gnuplot eps terminal output
% \end{figure}
% 
% \texttt{eqref} for nice eq. referencing  \eqref{sJInt}
% An example of \texttt{siunitx} package 
% for adding tolerances $A = \SI{12.8(4)}{s}$


\section{Diskuse}

%
% ----------------------------------------------------------------------
%  Závěr
% ----------------------------------------------------------------------
			
\section{Závěr}
