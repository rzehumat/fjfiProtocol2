% ----------------------------------------------------------------------
%  Identifikace protokolu (příkazy lze použít v celém dokumentu)
% ----------------------------------------------------------------------

%  Nastaví autora, název, datum, skupinu měření apod. 
\newcommand{\Institute}{FJFI~ČVUT~v~Praze}

% Examples -- subject
%\newcommand{\Subject}{Základy fyzikálních měření}
%\newcommand{\Subject}{Experimentální reaktorová fyzika}  %odkomentujte dle potřeby

%%%%%%%%%%%%%%%%%%%%%%%%%%%%%


%  Máte-li více spoluměřících než jednoho, vložte jen jejich příjmení
\newcommand{\Author}{}
\newcommand{\Coauthor}{} 
\newcommand{\Group}{} %číslo skupiny v rámci praktika, nikoli kruh
\newcommand{\Labdate}{} %den, kdy chodíte na praktika, nikoli obor

%  Tato část bude v každém protokolu jiná, nezapomeňte upravit!
\newcommand{\Title}{}
% \newcommand{\Labdate}{ -- } %datum měření, nikoli datum odevzdání

% I guess that is not the value one would like to share
% \newcommand{\Worktime}{5 h} %jak dlouho vám trvalo vypracování protokolu




% ----------------------------------------------------------------------
%  Vlastní příkazy
% ----------------------------------------------------------------------


%  Matematika
\newcommand{\ee}{\mathrm{e}} %eulerovo číslo
\newcommand{\ii}{\mathrm{i}} %imaginární jednotka

% Jednotky
\newcommand{\unit}[1]{\,\mathrm{#1}} %jednotky zadávejte pomocí tohoto příkazu
\renewcommand{\deg}{\ensuremath{\mathring{\;}}} %symbol stupně
\newcommand{\celsius}{\ensuremath{\deg\mathrm{C}}} %stupně celsia

%(hodnota plus mínus chyba) jednotka
\newcommand{\hodn}[3]{(#1 \pm #2)\unit{#3}} 

%veličina [jednotka] do hlavičky tabulky
\newcommand{\tabh}[2]{\ensuremath{#1\,[\mathrm{#2}]}} 



